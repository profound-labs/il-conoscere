\chapter{Buddha, Dhamma, Saṅgha}

Quando le persone ci chiedono: ``Cosa è necessario fare per diventare
buddhisti?'', rispondiamo che abbiamo preso rifugio nel Buddha, nel
Dhamma, nel Saṅgha. E per prendere rifugio recitiamo una formula in
lingua pāḷi:

\begin{quote}
\itshape
Buddhaṃ saranaṃ gacchāmi

\vin Prendo rifugio nel Buddha

Dhammaṃ saranaṃ gacchāmi

\vin Prendo rifugio nel Dhamma

Saṅghaṃ saranaṃ gacchāmi

\vin Prendo rifugio nel Saṅgha.
\end{quote}

Man mano che procediamo nella pratica, e cominciamo a realizzare la
profondità degli insegnamenti buddhisti, prendere questi rifugi diventa
una vera gioia - anche solo recitarli è fonte di ispirazione per la
mente. Sono monaco da ventidue anni, eppure mi piace ancora recitare
``\emph{Buddhaṃ saranaṃ gacchāmi}'', a dire il vero ancor più di quanto
non mi piacesse ventuno anni fa. All'epoca, infatti, queste parole non
avevano per me alcun significato, le intonavo perché dovevo farlo,
perché faceva parte della tradizione. Prendere rifugio nel Buddha
semplicemente a livello verbale non vuole affatto dire che si stia
prendendo rifugio in qualcosa: anche un pappagallo potrebbe essere
addestrato a dire ``\emph{Buddhaṃ saranaṃ gacchāmi}'', e questo avrebbe
per lui altrettanto poco significato quanto lo ha per molti buddhisti.
Queste parole sono intese perché ci si rifletta sopra, perché le si
osservi e ci si interroghi sul loro significato: cosa vuol dire
``rifugio'', cosa vuol dire ``Buddha''? Quando diciamo: ``Prendo rifugio
nel Buddha'', cosa intendiamo? Come possiamo utilizzare tutto questo in
modo che non sia un insieme ripetuto di sillabe prive di senso, ma
qualcosa che ci aiuta davvero a ricordare, a darci una direzione, ad
accrescere la nostra devozione e dedizione al sentiero del Buddha?

La parola ``Buddha'' è una parola meravigliosa, significa ``colui che
conosce'', e il primo rifugio è nel Buddha in quanto personificazione
della saggezza. Una saggezza non personificata risulta per noi troppo
astratta: non siamo capaci di concepire una saggezza priva di corpo,
priva di anima, e così, poiché la saggezza sembra avere sempre una
qualità personale, utilizzare il Buddha come suo simbolo è molto utile.

Possiamo usare la parola Buddha per riferirci a Gotama, il fondatore di
quello che oggi è noto come buddhismo, il saggio storico che raggiunse
il \emph{Parinibbāna} in India 2.500 anni fa, colui che insegnò le
Quattro Nobili Verità e l'Ottuplice Sentiero, insegnamenti da cui
traiamo beneficio ancor oggi. Tuttavia, quando prendiamo rifugio nel
Buddha non vuol dire che stiamo prendendo rifugio in un qualche profeta
storico, ma in ciò che vi è di saggio nell'universo, nelle nostre menti,
in ciò che non è separato da noi ma che è più reale di qualsiasi cosa si
possa concepire con la mente o di cui si possa fare esperienza
attraverso i sensi. Senza una ``saggezza-di-Buddha'', sarebbe
assolutamente impossibile una vita di una qualche durata nell'universo;
è la ``saggezza-di-Buddha'' che protegge. Noi la chiamiamo
``saggezza-di-Buddha'', altri possono utilizzare termini diversi -- in
fin dei conti si tratta solo di parole, e noi utilizziamo quelle della
nostra tradizione. Non litigheremo sulle parole pāḷi, sulle parole
sanscrite, o su quelle ebraiche, greche, latine, inglesi o di altro tipo
- stiamo solo usando l'espressione ``saggezza-di-Buddha'' come simbolo
convenzionale che ci aiuti a ricordare di essere saggi, di essere
vigili, di essere svegli.

Nel nord-est della Thailandia molti monaci della foresta usano la parola
``Buddho'' come oggetto di meditazione. La utilizzano come una sorta di
\emph{koan}. Prima calmano la mente seguendo le inspirazioni e le
espirazioni usando le sillabe BUD-DHO, poi iniziano a contemplare:
``Cosa è Buddho, `colui che conosce'? Cosa è il conoscere?''.

Quando andavo in \emph{tudong} (errare a piedi) percorrendo il nord-est
della Thailandia mi piaceva fermarmi presso il monastero di Ajahn Fun.
Ajahn Fun era un monaco molto amato e profondamente rispettato,
l'insegnante della Famiglia Reale, ed era così popolare che riceveva
ospiti in continuazione. Io mi sedevo presso la sua \emph{kuṭī}
(capanna) e lo stavo ad ascoltare mentre teneva alcuni dei suoi
straordinari discorsi di Dhamma, tutti incentrati sul tema ``Buddho'' --
da quanto potevo capire, insegnava solo questo. Riusciva a trasformare
questa pratica in una meditazione davvero profonda, adatta sia a un
contadino analfabeta che a un aristocratico thailandese elegante ed
educato all'occidentale. La parte principale del suo insegnamento non
consisteva nel ripetere solo meccanicamente ``Buddho'', ma nel
riflettere e investigare, nel risvegliare la mente a osservare in
profondità ``Buddho'', ``colui che conosce'', investigando realmente il
suo inizio e la sua fine, sopra e sotto, così che la propria attenzione
vi aderisse completamente. Quando si faceva così, la parola ``Buddho''
diventava qualcosa che risuonava nella mente. La si investigava, la si
osservava, la si esaminava, prima di pronunciarla e dopo averla
pronunciata, e a un certo punto si cominciava ad ascoltarla, e a sentire
al di là del suono, fino a quando si sentiva il silenzio.

Un rifugio è un luogo in cui si è al sicuro, e così quando delle persone
superstiziose si recavano dal mio maestro Ajahn Chah chiedendogli
medaglioni avvolti da incantesimi o piccoli talismani che li
proteggessero da proiettili, coltelli, fantasmi e così via, lui
rispondeva: ``Perché volete queste cose? La sola vera protezione è
prendere rifugio nel Buddha. Prendere rifugio nel Buddha è
sufficiente''. Ma di solito la loro fede nel Buddha non era altrettanto
forte di quella in questi sciocchi medaglioni. Loro volevano qualcosa di
bronzo o di creta, timbrato e benedetto. Questo vuol dire prendere
rifugio nel bronzo e nella creta, prendere rifugio nella superstizione,
prendere rifugio in ciò che è davvero insicuro e non ci può essere
realmente d'aiuto.

Al giorno d'oggi nella Gran Bretagna moderna le persone sono in genere
più sofisticate, non prendono rifugio negli amuleti magici, prendono
rifugio in cose come la Westminster~Bank - ma anche questo vuol dire
prendere rifugio in qualcosa che non offre alcuna sicurezza. Prendere
rifugio nel Buddha, nella saggezza, significa avere un luogo sicuro.
Quando c'è la saggezza, quando agiamo saggiamente e viviamo saggiamente,
siamo realmente al sicuro. Le condizioni intorno a noi potrebbero
cambiare, non abbiamo alcuna garanzia su quale sarà il tenore di vita, o
se la Westminster Bank supererà questo decennio. Il futuro rimane
sconosciuto e misterioso, ma nel presente, prendendo rifugio nel Buddha,
abbiamo la presenza mentale che ci consente di riflettere sulla vita e
imparare da essa mentre la viviamo.

Saggezza non vuol dire conoscere tante cose sul mondo; non abbiamo
bisogno di andare all'università e raccogliere informazioni sul mondo
per essere saggi. Saggezza vuol dire conoscere la natura delle
condizioni mentre ne facciamo esperienza. Non consiste nel farsi
catturare e assorbire per forza d'abitudine dalle condizioni dei nostri
corpi e delle nostre menti, reagendo condizionati dalla paura, dalla
preoccupazione, dal dubbio, dall'avidità e così via, ma vuol dire usare
quel ``Buddho'', ``colui che conosce'', per osservare che queste
condizioni sono mutevoli. È proprio l'atto del conoscere quel
cambiamento ciò che chiamiamo Buddha e in cui prendiamo rifugio. Non
abbiamo la pretesa che quel Buddha sia ``me'' o ``mio''. Non diciamo
``Io sono Buddha'', ma piuttosto ``Prendo rifugio nel Buddha''. È un
modo per sottomettersi umilmente a quella saggezza, essendo consapevoli,
essendo svegli.

Benché in un certo senso prendere rifugio sia qualcosa che facciamo
continuamente, la formula pāḷi che utilizziamo è un richiamo per la
memoria, perché dimentichiamo, perché prendiamo abitualmente rifugio
nella preoccupazione, nel dubbio, nella paura, nella rabbia,
nell'avidità e così via. L'immagine del Buddha ha una funzione simile;
quando ci inchiniamo di fronte ad essa non immaginiamo che sia qualcosa
di diverso da un'immagine di bronzo, un simbolo. È un modo di riflettere
che ci rende un po' più consapevoli del Buddha, del nostro rifugio nel
Buddha, nel Dharma, nel Saṅgha. L'immagine del Buddha siede con grande
calma e dignità, non in trance ma pienamente vigile, con uno sguardo
colmo di consapevolezza e gentilezza, non scalfita dalle condizioni
mutevoli intorno ad essa. Sebbene l'immagine sia fatta di bronzo, mentre
noi abbiamo questi corpi di carne e sangue, e quindi per noi è molto più
difficile, pur tuttavia è un richiamo. Alcune persone assumono un
atteggiamento bigotto rispetto alle immagini del Buddha, ma qui in
Occidente non ho trovato che costituiscano un pericolo. I veri idoli in
cui crediamo e che veneriamo, e che ci ingannano continuamente, sono i
nostri pensieri, le nostre opinioni e convinzioni, i nostri amori e odi,
la nostra arroganza e presunzione.

Il secondo rifugio è nel Dhamma, nella verità ultima o realtà ultima. Il
Dhamma è impersonale; non cerchiamo in alcun modo di personificarlo, di
trasformarlo in una qualche sorta di deità personale. Quando recitiamo
in pāḷi il verso sul Dhamma, diciamo che esso è ``\emph{sandihiṭṭhiko
akāliko ehipassiko opanayiko paccattaṃ veditabbo viññūhi}''. Dato che il
Dhamma non ha attributi personali, non possiamo nemmeno dire che sia
buono o cattivo, o dotato di qualsiasi altra qualità superlativa o di
paragone: esso è al di là delle concezioni dualistiche della mente.

Perciò quando descriviamo il Dhamma o vi alludiamo lo facciamo
attraverso parole come ``\emph{sandihiṭṭhiko}'', che vuol dire
immanente, qui-e-ora. Questo ci riporta al presente; avvertiamo un senso
di immediatezza, di attualità. Si potrebbe pensare che il Dhamma sia un
qualcosa che è ``là fuori'', qualcosa da ricercare altrove, ma
\emph{sandihiṭṭhikodhamma} vuol dire che è immanente, qui-e-ora.

\emph{Akālikadhamma} significa che il Dhamma non è vincolato da alcuna
condizione temporale. La parola \emph{akāla} significa senza tempo. La
nostra mente concettuale non è in grado di concepire qualcosa che sia
senza tempo, perché le nostre concezioni e percezioni sono condizioni
basate sul tempo, però ciò che possiamo dire è che il Dhamma è
\emph{akāla}, non condizionato dal tempo.

\emph{Ehipassikadhamma} è un richiamo ad andare a vedere, a volgersi o
dirigersi verso il Dhamma. Vuol dire guardare, essere consapevoli. Non è
che preghiamo il Dhamma di venire, o aspettiamo che ci batta sulla
spalla; dobbiamo impegnarci attivamente. È come nel detto di Gesù:
``Bussate e vi sarà aperto''. \emph{Ehipassiko} vuol dire che dobbiamo
mettere in atto lo sforzo di volgerci verso la verità.

\emph{Opanayiko} vuol dire che porta verso l'interiorità, verso la pace
all'interno della mente. Il Dhamma non ci spinge verso ciò che è
affascinante o eccitante, verso le romanticherie o l'avventura, ma
conduce al Nibbāna, alla calma, al silenzio.

\emph{Paccattaṃ veditabbo viññūhi} vuol dire che possiamo conoscere il
Dhamma solamente attraverso l'esperienza diretta. È come il sapore del
miele, se lo assaggia qualcun altro continueremo a non conoscerne il
sapore. Potremmo conoscerne la formula chimica, o essere in grado di
recitare tutto ciò che di poetico è stato scritto sul miele, ma solo
quando lo assaggeremo personalmente sapremo davvero com'è. È lo stesso
con il Dhamma: dobbiamo assaggiarlo, dobbiamo conoscerlo direttamente.

Prendere rifugio nel Dhamma vuol dire prendere un altro rifugio sicuro.
Non si prende rifugio in una filosofia o in concetti intellettuali, in
teorie, idee, dottrine o credenze di qualche tipo. Non ci rifugiamo in
una credenza nel Dhamma, o in Dio, o in qualche forza là fuori nello
spazio, oppure in qualcosa che è ``al di là'' o è separato, qualcosa che
dovremo incontrare prima o poi nel futuro. Le descrizioni del Dhamma ci
tengono ancorati al presente, nel qui e ora libero da vincoli temporali.
Prendere rifugio è un modo di riflessione immediato, immanente nella
mente; non consiste nel ripetere come pappagalli ``\emph{Dhammaṃ saraṇaṃ
gacchāmi}'', pensando ``I buddhisti dicono così per cui lo devo dire
anch'io''. Noi ci volgiamo verso il Dhamma, siamo consapevoli adesso,
prendiamo rifugio nel Dhamma adesso, come un'azione immediata, un
riflesso immediato dell'essere il Dhamma, essere quella stessa verità.

La nostra mente, nel suo elaborare continuamente concetti, tende sempre
a fuorviarci, portandoci nel divenire. Pensiamo: ``Praticherò la
meditazione così un giorno mi illuminerò. Prenderò i Tre Rifugi per
diventare buddhista. Voglio diventare saggio. Voglio sfuggire alla
sofferenza e all'ignoranza ed essere diverso''. Questa è la mente
concettuale, la mente desiderante, la mente che ci trae sempre in
inganno. Invece di pensare continuamente secondo la prospettiva del
divenire in futuro qualcosa, prendiamo rifugio nell'essere Dhamma nel
presente.

L'impersonalità del Dhamma costituisce un problema per molte persone,
perché la religione devozionale tende a personificare tutto, e chi viene
da questo tipo di tradizioni non si sente a suo agio se non in una
qualche forma di relazione personale. Ricordo che un giorno un
missionario cattolico francese venne nel nostro monastero per praticare
la meditazione. Si trovò un po' in difficoltà con l'approccio buddhista,
perché disse che era come una ``gelida operazione chirurgica'', non
c'era una relazione personale con Dio. Non si può avere una relazione
personale con il Dhamma, non si può dire ``Ama il Dhamma!'', o ``Il
Dhamma mi ama!''; non ce n'è alcun bisogno. Abbiamo bisogno di una
relazione personale solo con ciò che non siamo, come una madre o un
padre, un marito o una moglie, qualcosa di separato da noi. Ma non
abbiamo bisogno di prendere rifugio nel papà o nella mamma, in qualcuno
che ci protegga e ci ami, e che ci dica carezzandoci la testa: ``Ti amo
qualsiasi cosa tu faccia. Andrà tutto bene''. Il Buddha-Dhamma è un
rifugio che rende maturi, è una pratica religiosa completamente matura e
sana, in cui non andiamo più alla ricerca di un padre o di una madre,
perché non abbiamo più bisogno di diventare qualcosa di diverso. Non
abbiamo più bisogno di essere amati e protetti da qualcuno, perché
possiamo amare e proteggere gli altri, e questo è tutto ciò che conta.
Non dobbiamo più chiedere o implorare nulla dagli altri, che si tratti
di persone o persino di una divinità o forza che percepiamo come
separata da noi, a cui dobbiamo rivolgere preghiere e chiedere di
guidarci. Rinunciamo a tutti i nostri tentativi di concepire il Dhamma
in questo o quel modo, in qualsiasi modo, e lasciamo andare il nostro
desiderio di avere una relazione personale con la verità. Dobbiamo
essere quella verità, qui e ora. Essere quella verità, prendere quel
rifugio, richiede che ci si risvegli nell'immediato, richiede che si sia
saggi adesso, che si sia Buddha, che si sia Dhamma nel presente.

Il terzo rifugio è il Saṅgha, che vuol dire un gruppo. ``Saṅgha'' può
essere il \emph{Bhikku-Saṅgha}, l'ordine dei monaci, oppure
l'\emph{Ariya-Saṅgha}, il gruppo degli Esseri Nobili, tutti
coloro che conducono una vita virtuosa, facendo il bene e astenendosi
dal fare il male con azioni o parole. Prendere rifugio nel Saṅgha con la
frase ``\emph{Saṅghaṃ saranaṃ gacchāmi}'' significa che prendiamo
rifugio nella virtù, in ciò che è buono, virtuoso, gentile,
compassionevole e generoso. Non prendiamo rifugio in quelle parti della
nostra mente che sono meschine, malevole, crudeli, egoiste, invidiose,
piene di odio e di rabbia, anche se non c'è dubbio che questo è ciò che
spesso tendiamo a fare per incuria, quando non riflettiamo e non siamo
vigili, e quindi reagiamo semplicemente alle condizioni. Prendere
rifugio nel Saṅgha vuol dire, a livello convenzionale, fare il bene e
astenersi dal fare il male con azioni o parole.

Tutti noi abbiamo intenzioni e pensieri sia buoni che cattivi. I
saṅkhāra (i fenomeni condizionati) sono così: alcuni sono buoni e altri
no, alcuni sono neutri, alcuni sono meravigliosi e altri molto
sgradevoli. Nel mondo le condizioni sono mutevoli. Non possiamo pensare
solo i pensieri migliori e più elevati, e provare soltanto i sentimenti
migliori e più gentili; i pensieri e i sentimenti buoni e cattivi vanno
e vengono, ma noi prendiamo rifugio nella virtù piuttosto che nell'odio.
Prendiamo rifugio in ciò che in tutti noi ha l'intenzione di fare il
bene, che è compassionevole, gentile e amorevole verso noi stessi e
verso gli altri.

Il rifugio del Saṅgha è dunque un rifugio molto pratico per vivere
quotidianamente in questa forma umana, in questo corpo, in relazione ai
corpi degli altri esseri e al mondo fisico in cui viviamo. Quando
prendiamo questo rifugio evitiamo di agire in qualunque modo possa
essere causa di divisione, disarmonia, crudeltà, meschinità o cattiveria
verso qualsiasi essere vivente, inclusi noi stessi, il nostro corpo e la
nostra mente. Questo vuol dire essere ``\emph{supaṭipanno}'', uno che
pratica bene.

Quando siamo consapevoli e attenti, quando riflettiamo e osserviamo,
cominciamo a vedere che agire spinti da impulsi crudeli ed egoistici
arreca soltanto danno e infelicità a noi stessi e agli altri. Non ci
vuole chissà quale potere di osservazione per constatarlo. Chi ha
incontrato qualche criminale, persone che hanno agito egoisticamente e
con malvagità, troverà che sono costantemente impauriti, ossessionati,
paranoici, sospettosi. Hanno bisogno di bere molto, di assumere droghe,
di tenersi occupati in ogni sorta di attività, perché vivere con se
stessi è orribile. Passare cinque minuti da soli con se stessi senza
droghe o alcol, o senza fare qualcosa, sembrerebbe loro un inferno senza
fine, perché a livello mentale il risultato kammico della malvagità è
terribile. Anche se non venissero mai catturati dalla polizia o mandati
in prigione, non bisogna pensare che la farebbero franca. In realtà, a
volte la cosa più gentile da fare è proprio metterli in prigione e
punirli: li fa sentire meglio. Io non sono mai stato un criminale, ma
nel corso della mia vita sono riuscito a dire qualche bugia e a fare
qualche azione meschina e cattiva, e i risultati sono sempre stati
spiacevoli. Ancor oggi, quando ci ripenso, non sono ricordi gradevoli,
non sono cose che vorrei raccontare a tutti, né mi procurano gioia
quando mi tornano in mente.

Quando meditiamo realizziamo che dobbiamo essere completamente
responsabili per il modo in cui viviamo. Non possiamo in alcun modo
incolpare qualcun altro di qualcosa. Prima che cominciassi a meditare
ero solito incolpare la gente e la società: ``Se solo i miei genitori
fossero stati degli \emph{arahant} pieni di saggezza e illuminati, io
starei bene. Se solo gli Stati Uniti avessero un governo davvero saggio
e compassionevole che non facesse mai errori, mi desse sostegno e mi
apprezzasse in pieno. Se solo i miei amici fossero saggi e
incoraggianti, e gli insegnanti saggi, generosi e gentili. Se tutti
intorno a me fossero perfetti, se la società fosse perfetta, se il mondo
fosse saggio e perfetto, allora io non avrei tutti questi problemi. Ma
hanno tutti fallito nei miei confronti''. I miei genitori avevano alcuni
difetti, e in realtà hanno fatto alcuni errori, ma quando ora ripenso al
passato, non ne avevano fatti nemmeno troppi. All'epoca, quando cercavo
di incolpare gli altri, e mi davo da fare per trovare le colpe dei miei
genitori, dovevo davvero impegnarmi per riuscirci. La mia generazione
era molto brava a incolpare di tutto gli Stati Uniti, e non era affatto
difficile, perché gli Stati Uniti fanno molti errori. Ma quando
meditiamo non possiamo più continuare a mentire così a noi stessi. Ci
rendiamo improvvisamente conto che non importa ciò che qualcun altro ha
fatto, o quanto sia ingiusta la società, o come siano stati i nostri
genitori, non possiamo più passare il resto della nostra vita a
incolpare qualcuno altro -- è un'assoluta perdita di tempo. Dobbiamo
accettare la totale responsabilità per la nostra vita, e viverla. Se
anche davvero i nostri genitori fossero stati tremendi, se fossimo
cresciuti in una società spaventosa che non ci avesse offerto alcuna
opportunità, non importa ugualmente. Non possiamo incolpare nessun altro
per la nostra sofferenza attuale se non noi stessi, la nostra ignoranza,
il nostro egoismo, la nostra presunzione.

Nella crocifissione di Gesù possiamo vedere un esempio straordinario di
un uomo in preda al dolore, denudato, deriso, totalmente umiliato e poi
giustiziato pubblicamente nel più orribile e atroce dei modi, il quale
però non ha mai incolpato nessuno: ``Padre, perdona loro perché non
sanno quello che fanno''. Questo è un segno di saggezza, vuol dire che
anche se le persone ci stanno crocifiggendo, inchiodando alla croce,
flagellando, umiliando in ogni modo, è
la nostra avversione, la nostra autocommiserazione e auto-centratura che
sono il problema, la vera sofferenza. Non è nemmeno il dolore fisico che
fa davvero soffrire, è l'avversione. Ora, se Gesù Cristo avesse detto:
``Vi maledico per ciò che mi state facendo!'', sarebbe stato solo un
altro criminale e sarebbe stato dimenticato dopo pochi giorni.

Riflettete sul perché tendiamo facilmente a incolpare gli altri per la
nostra sofferenza. Potrebbe essere anche giustificabile, perché forse
c'è qualcuno che ci sta maltrattando, o sfruttando, o non ci sta
capendo, o ci sta facendo cose tremende. Non neghiamo tutto ciò, ma non
ce ne curiamo più. Perdoniamo, lasciamo andare questi ricordi, perché
prendere rifugio nel Saṅgha vuol dire, qui e ora, fare il bene e
astenerci dal fare il male con azioni o parole.

Che possiate dunque riflettere su tutto ciò, e vedere davvero il Buddha,
il Dhamma e il Saṅgha come un rifugio. Considerateli come opportunità
per riflettere e contemplare. Non si tratta di credere nel Buddha, nel
Dhamma, nel Saṅgha - non si tratta di avere fede in concetti - ma
piuttosto di usarli come simboli per la consapevolezza, per risvegliare
la mente qui e ora; per essere qui e ora.
