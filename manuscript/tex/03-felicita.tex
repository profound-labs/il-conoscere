\chapter{Felicità, Infelicità e Nibbāna}

Il fine ultimo della meditazione buddhista è il Nibbāna. Ci orientiamo
verso la pace del Nibbāna, allontanandoci dalle complessità della sfera
sensoriale, dai cicli infiniti dell'abitudine. Il Nibbāna è una meta
realizzabile in questa vita, non è necessario aspettare fino al momento
della morte per sapere se è reale.

I sensi e il mondo sensoriale sono governati da nascita e morte.
Prendiamo ad esempio la vista: dipende da così tanti fattori, se è
giorno o notte, se gli occhi sono sani oppure no, e così via. Eppure ci
attacchiamo moltissimo ai colori, alle figure e alle forme che
percepiamo con gli occhi, e ci identifichiamo con essi. Poi ci sono le
orecchie e i suoni: quando udiamo dei suoni piacevoli cerchiamo di
trattenerli, e quando udiamo dei suoni spiacevoli cerchiamo di
allontanarcene. Con l'olfatto ricerchiamo il piacere di profumi e odori
gradevoli, cercando di sottrarci a quelli sgradevoli.
\protect\hypertarget{__DdeLink__546_1053081209}{}{}È lo stesso con i
sapori: ricerchiamo i sapori deliziosi e facciamo di tutto per evitare
quelli cattivi. E con il tatto: quanta parte della nostra vita
trascorriamo cercando di sfuggire al disagio e al dolore fisico, alla
ricerca di sensazioni fisiche appaganti? Infine c'è il pensiero, la
coscienza discriminante: può arrecarci molto piacere o molta angoscia.

Questi sono i sensi, il mondo sensoriale. È il mondo composto di nascita
e morte. La sua stessa natura è \emph{dukkha}, è imperfetto e
insoddisfacente. Non potremo mai trovare felicità, gioia o pace perfette
nel mondo sensoriale, che porterà sempre ad afflizione e morte. Il mondo
dei sensi è insoddisfacente, dunque ci fa soffrire solo quando pensiamo
ci debba soddisfare. Ci fa soffrire quando pretendiamo che ci dia più di
quanto possa dare, cose come sicurezza e felicità durature, amore e
certezze durature, sperando in una vita fatta di solo piacere e senza
sofferenza. ``Se solo potessimo fare a meno di infermità e malanni e
riuscissimo a sconfiggere la vecchiaia!''.

Mi ricordo che vent'anni fa negli Stati Uniti la gente era fiduciosa che
la scienza moderna sarebbe stata in grado di eliminare tutte le
malattie. Si diceva: ``Tutte le malattie mentali sono dovute a squilibri
chimici. Se soltanto riusciamo a trovare le giuste combinazioni chimiche
e le iniettiamo nel corpo la schizofrenia scomparirà''. Non ci sarebbero
più stati mal di testa o mal di schiena. Avremmo gradualmente sostituito
tutti i nostri organi interni con dei begli organi di plastica. Ho
persino letto in una rivista medica australiana come sperassero di
sconfiggere la vecchiaia! Con la crescita della popolazione mondiale
avremmo continuato ad avere sempre più bambini, e nessuno sarebbe
invecchiato o morto. Pensate che caos sarebbe stato!

Il mondo sensoriale è insoddisfacente ed è così che dev'essere. Quando
costituisce un attaccamento, ci porta all'afflizione -- perché
attaccamento vuol dire che vogliamo che sia soddisfacente, vogliamo che
ci appaghi che ci renda contenti, felici e al sicuro. Ma osservate la
natura della felicità: quanto a lungo si può rimanere felici? Cos'è la
felicità? Si potrebbe supporre che sia come ti senti quando ottieni ciò
che desideri. Qualcuno dice qualcosa che ti è gradito, e ti senti
felice. Qualcuno dice qualcosa che approvi, e ti senti felice. Il sole
splende e ti senti felice. Qualcuno prepara del buon cibo e te lo serve,
e sei felice. Ma quanto a lungo puoi restare felice? Dobbiamo sempre
dipendere dal fatto che il sole splenda? In Inghilterra il clima è molto
variabile: in questo paese una felicità che dipendesse dal fatto che il
sole splenda sarebbe ovviamente molto impermanente e insoddisfacente!

Infelicità è non ottenere ciò che vogliamo: volere la presenza del sole
quando fa freddo, è umido e piove; quando le persone fanno cose che non
approviamo; se abbiamo di fronte del cibo che non è delizioso, e così
via. La vita diventa noiosa e opprimente quando siamo infelici con ciò
che ci offre. Così felicità e infelicità dipendono molto dal fatto che
si ottenga ciò che si vuole, o che si ottenga ciò che non si vuole.

Ma per la maggior parte delle persone lo scopo della loro vita è la
felicità. La Costituzione Americana mi sembra che parli del ``diritto al
perseguimento della felicità''. Ottenere ciò che vogliamo, ciò di cui
riteniamo avere diritto, diventa lo scopo della nostra vita. Ma la
felicità porta sempre all'infelicità, perché è impermanente. Quanto a
lungo si può davvero essere felici? Cercare di organizzare, controllare
e manipolare le condizioni in modo da ottenere sempre ciò che
desideriamo, udire sempre ciò che vogliamo udire, vedere sempre ciò che
vogliamo vedere, così da non dover mai provare infelicità o afflizione,
è un'impresa impossibile. È impossibile, no? La felicità è
insoddisfacente, è \emph{dukkha}. Non è qualcosa da cui dipendere, o da
trasformare nello scopo della propria esistenza. La felicità sarà sempre
deludente, perché dura così poco e poi è seguita dall'infelicità.
Dipende sempre da così tante condizioni. Ci sentiamo felici quando siamo
in buona salute, ma i nostri corpi umani sono soggetti a rapidi
cambiamenti e quella salute può svanire molto velocemente. Allora ci
sentiamo terribilmente infelici, perché siamo malati e abbiamo perso il
piacere che provavamo quando ci sentivamo pieni di energia e vigore.

Perciò lo scopo del Buddhismo non è la felicità, perché comprendiamo che
la felicità è insoddisfacente. Il vero obiettivo si trova lontano dal
mondo sensoriale. Non si tratta di un rifiuto del mondo sensoriale, ma
avendolo compreso così bene non lo ricerchiamo più come un obiettivo in
sé. Non ci aspettiamo più che il mondo sensoriale ci appaghi. Non
pretendiamo più che la coscienza sensoriale sia altro che una condizione
esistente da utilizzare in modo abile secondo il tempo e il luogo. Non
proviamo più attaccamento nei suoi confronti, esigendo che l'impatto dei
sensi sia sempre piacevole, affliggendoci e angosciandoci quando è
spiacevole. Il Nibbāna non è uno stato di assenza, una trance in cui ci
si annulla completamente. Non è una inesistenza o un annichilimento: è
come uno spazio. È entrare nello spazio della mente in cui non c'è più
attaccamento, in cui non si è più tratti in inganno dall'apparenza delle
cose. Non si pretende più nulla dal mondo sensoriale, lo si riconosce
semplicemente mentre sorge e svanisce.

Essere nati nella condizione umana vuol dire che dobbiamo
inevitabilmente sperimentare vecchiaia, malattia e morte. Un giorno una
giovane donna venne al nostro monastero in Inghilterra con il suo bimbo
piccolo il quale da una settimana si era ammalato, squassato da
un'orribile tosse. La donna sembrava terribilmente depressa e
angosciata. Mentre era seduta nella sala d'accoglienza il bambino in
braccio alla madre diventò paonazzo, cominciando a strillare e tossire
orribilmente. La donna disse: ``Venerabile Sumedho, perché deve soffrire
così? Non ha mai fatto del male a nessuno, non ha mai fatto qualcosa di
sbagliato. Allora perché? Cos'ha fatto in una vita precedente per dovere
soffrire così?''. Stava soffrendo perché era nato! Se non fosse nato non
avrebbe dovuto soffrire. Quando nasciamo dobbiamo aspettarci queste
situazioni. Avere un corpo umano vuol dire che dobbiamo sperimentare
malattia, dolore, vecchiaia e morte. Questa è una riflessione
importante. Possiamo presumere che forse in una vita precedente quel
bambino si divertisse a soffocare cani e gatti, o qualcosa di simile, e
che in questa vita dovesse pagare per quello che aveva compiuto, ma
sarebbe solo una congettura che non ci aiuterebbe veramente. Ciò che
possiamo sapere è che si tratta del risultato kammico per il fatto di
essere nati. Ognuno di noi deve inevitabilmente sperimentare malattia e
sofferenza, fame e sete, il processo di invecchiamento dei nostri corpi
e la morte. È la legge del kamma. Ciò che ha inizio deve avere una fine;
ciò che nasce deve morire; ciò che è unito si deve separare.

Questo non vuol dire essere pessimisti: stiamo osservando le cose per
quello che sono, dunque non ci aspettiamo che la vita sia diversa da
quella che è. In questo modo la possiamo affrontare, sopportandola
quando è difficile e rallegrandocene quando è piacevole. Se comprendiamo
la vita possiamo goderne, senza esserne le vittime impotenti. C'è così
tanta sofferenza nell'esistenza umana perché pretendiamo che la vita sia
diversa da quella che è! Abbiamo tutte queste idee romantiche, che
incontreremo la persona giusta, ci innamoreremo e vivremo per sempre
felici e contenti, non litigheremo mai e avremo una relazione
meravigliosa. Ma\ldots{} e la morte? Allora pensiamo: ``Beh, forse
moriremo contemporaneamente''. Questa è una speranza, dico bene? Si
spera, e poi ci si dispera quando la persona amata muore prima di te, o
scappa con lo spazzino o il commesso viaggiatore.

Si può imparare moltissimo dai bambini piccoli, perché non nascondono
ciò che provano, ma esprimono soltanto ciò che sentono in quel momento:
quando sono infelici cominciano a piangere, e quando sono felici ridono.
Qualche tempo fa sono andato assieme a un laico a casa sua. Quando siamo
arrivati, la sua figlioletta era felicissima di vederlo. Poi lui le ha
detto: ``Devo portare il Venerabile Sumedho all'Università del Sussex
per un discorso''. Mentre stavamo uscendo dalla porta la bimba diventò
tutta rossa in volto e cominciò a strillare disperata, così il padre le
disse: ``Va tutto bene, sarò di ritorno entro un'ora'', ma la bambina
non era abbastanza grande da capire: ``Tornerò entro un'ora''.
L'immediatezza della separazione dalla persona amata significava
angoscia immediata.

Osservate quante volte nella nostra vita soffriamo in questo modo nel
doverci separare da qualcosa che ci piace o da qualcuno che amiamo, o
nel dover lasciare un posto in cui ci piace stare. Quando si è veramente
consapevoli si può riconoscere quella sensazione del non volersi
separare, e quella sofferenza. Da adulti, se sappiamo che possiamo
tornare, possiamo lasciarla andare subito, sapendo però che è pur sempre
lì. Dallo scorso novembre a marzo ho viaggiato in tutto il mondo e,
arrivando in un aeroporto, ho sempre trovato qualcuno che mi salutava
con un ``Salve!''. Poi, alcuni giorni dopo, c'era un ``Arrivederci!'', e
sempre quell'auspicio ``Ritorna'', e io dicevo ``Sì, tornerò''\ldots{}
impegnandomi per l'anno successivo a rifare la stessa cosa. Non siamo
capaci di dire ``Addio per sempre'' a qualcuno che ci piace, non è vero?
Diciamo ``Ci rivediamo'', ``Ti telefono'', ``Ti scriverò'', ``Al nostro
prossimo incontro''. Abbiamo tutte queste frasi per nascondere la
sensazione di tristezza e di separazione.

Nella meditazione stiamo osservando, notando semplicemente cos'è davvero
la tristezza. Non stiamo dicendo che non dovremmo sentirci tristi quando
ci separiamo da qualcuno che amiamo; è solo naturale sentirsi così,
giusto? Ma ora, come meditanti, cominciamo a essere testimoni della
tristezza in modo da comprenderla invece di reprimerla, immaginarla più
grande di quanto non sia, o semplicemente ignorarla.

In Inghilterra le persone tendono a reprimere la tristezza quando
qualcuno muore. Cercano di non piangere o mostrare la loro commozione,
non vogliono dare spettacolo, si danno un contegno. Poi quando iniziano
a meditare capita che si ritrovino improvvisamente a piangere sulla
morte di qualcuno scomparso quindici anni prima. Non avevano pianto
allora, e si ritrovano a farlo quindici anni dopo. Quando qualcuno muore
non vogliamo ammettere il nostro dolore facendone un dramma, perché
pensiamo che piangendo ci mostreremmo deboli, o saremmo di imbarazzo per
gli altri. Così tendiamo a reprimere e a trattenere, senza riconoscere
la natura delle cose per come sono realmente, senza riconoscere la
difficoltà della nostra condizione umana, e imparare da quella. Quando
meditiamo permettiamo alla mente di aprirsi, lasciando che tutto ciò che
è stato soppresso e represso divenga conscio, perché quando le cose
diventano consce trovano il modo di cessare invece di venire nuovamente
represse. Permettiamo che le cose facciano il loro corso fino a quando
non si estinguono, permettiamo che se ne vadano da sole invece di
respingerle. Di solito ci limitiamo ad allontanare da noi determinate
cose, rifiutando di accettarle o riconoscerle. Se ci sentiamo turbati o
irritati con qualcuno, se siamo annoiati o sorgono emozioni spiacevoli
ci mettiamo a guardare la bellezza di un fiore oppure il cielo, leggiamo
un libro, guardiamo la TV, facciamo qualcosa. Non siamo mai del tutto
consapevolmente annoiati o arrabbiati. Non riconosciamo la nostra
afflizione o la nostra delusione, perché possiamo sempre rifugiarci in
qualcos'altro. Possiamo sempre andare al frigorifero, mangiare dolci e
pasticcini, ascoltare la radio. È così facile immergersi nella musica,
lontano dalla noia e dall'afflizione, assorti in qualcosa che è
eccitante, interessante, tranquillizzante, o bello. Guardate quanto
siamo dipendenti dalla televisione o dalla lettura. Oggigiorno ci sono
così tanti libri che alla fine dovranno essere bruciati, ovunque libri
inutili, tutti scrivono qualcosa senza avere nulla che meriti di essere
detto. Le stelle del cinema d'oggi -- a dire il vero non molto gradevoli
- scrivono le proprie biografie e fanno soldi a palate. Poi ci sono le
rubriche di gossip: la gente fugge dalla noia della propria esistenza,
dal tedio e dallo scontento per la propria vita, leggendo i pettegolezzi
sulle stelle del cinema e i personaggi pubblici.

Non abbiamo mai veramente accettato la noia come stato consapevole. Non
appena la mente l'avverte cominciamo a cercare qualcosa di interessante,
qualcosa di piacevole. Ma nella meditazione permettiamo alla noia di
esistere. Ci permettiamo di essere annoiati in totale consapevolezza, ci
concediamo di essere pienamente depressi, scocciati, gelosi, arrabbiati,
disgustati. Tutte le esperienze odiose e sgradevoli che abbiamo represso
ed escluso dalla coscienza senza mai guardarle veramente le cominciamo
invece ad accoglierle nella coscienza, non più come problemi della
personalità, ma solo per compassione. Mossi da gentilezza e saggezza,
permettiamo che le cose seguano il loro corso naturale fino alla
cessazione, invece di continuare a perpetuarle nei soliti cicli
ripetitivi dell'abitudine. Se non troviamo il modo di lasciare che le
cose seguano il loro corso naturale, allora stiamo sempre controllando,
intrappolati in qualche sterile abitudine mentale. Quando siamo stanchi
e depressi non siamo in grado di apprezzare la bellezza delle cose,
perché non le vediamo mai veramente per quelle che sono davvero.

Mi ricordo un'esperienza che ho avuto durante il mio primo anno di
meditazione in Thailandia. Trascorsi quasi tutto l'anno da solo in una
piccola capanna, e i primi mesi furono davvero terribili. Continuavano a
venirmi in mente tutti i generi di cose: ossessioni, paure, terrore e
odio. Non avevo mai provato così tanto odio. Non mi ero mai considerato
una persona che odiava la gente, ma durante quei primi pochi mesi di
meditazione mi sembrava di odiare tutti. Non riuscivo a pensare nulla di
positivo su nessuno, tanta era l'avversione che stava affiorando alla
coscienza. Poi un pomeriggio cominciai ad avere questa strana visione (a
dire il vero pensai che stavo impazzendo): vidi delle persone che
uscivano camminando dal mio cervello. Vidi mia madre che usciva dal mio
cervello incamminandosi verso il vuoto e scomparendo nello spazio. Poi
fu il turno di mio padre e mia sorella. Vidi realmente queste visioni
che camminando mi uscivano dalla testa. Pensai ``Sono pazzo! Mi è
partito il cervello!'', però non era un'esperienza spiacevole.

Quando la mattina seguente mi svegliai e mi guardai intorno, tutto
quello che vidi mi sembrò meraviglioso. Tutto, anche i dettagli meno
belli, era meraviglioso. Ero sbalordito e incantato. La capanna era una
struttura primitiva, che nessuno avrebbe potuto definire bella, ma mi
sembrò un palazzo. Fuori, gli alberi spelacchiati mi apparvero come la
più incantevole delle foreste. I raggi del sole, penetrando dalla
finestra, andavano a colpire un piatto di plastica, e il piatto di
plastica mi sembrò stupendo! Quella sensazione di bellezza rimase con me
per circa una settimana, poi, riflettendoci sopra, compresi
improvvisamente che questo è il modo in cui le cose sono realmente
quando la mente è limpida. Fino a quel momento avevo guardato attraverso
una finestra sporca, e col passare degli anni mi ero così abituato a
quel sudiciume e a quella sporcizia sulla finestra che non me ne
accorgevo nemmeno più, avevo creduto che le cose fossero semplicemente
così.

Quando ci abituiamo a guardare attraverso una finestra sporca, tutto
sembra grigio, sudicio e brutto. La meditazione è un modo per pulire la
finestra, purificando la mente, permettendo che le cose emergano alla
coscienza e lasciandole andare. Poi con la facoltà della saggezza, la
saggezza-di-Buddha, osserviamo come sono realmente. Questo non vuol dire
essere attaccati alla bellezza, alla purezza mentale, ma vuol dire
comprendere davvero. Vuol dire riflettere saggiamente sul modo in cui
opera la natura, così da non esserne più indotti a crearci delle
abitudini per la vita a causa della nostra ignoranza.

La nascita significa vecchiaia, malattia e morte, ma questo riguarda il
corpo, non sei tu. Il corpo umano non è realmente tuo. Non importa quale
sia il tuo aspetto, se sei sano o malaticcio, se sei bello oppure no, se
sei nero, bianco o altro, è tutto non-sé. Questo è quello che intendiamo
con \emph{anattā}, che il corpo umano appartiene alla natura, segue le
regole della natura: nasce, cresce, invecchia e muore. Ora, questo
possiamo anche comprenderlo razionalmente, ma emotivamente abbiamo un
grandissimo attaccamento nei confronti del corpo. Quando meditiamo
cominciamo a vedere questo attaccamento. Non prendiamo la posizione per
cui non dovremmo provare attaccamento, dicendo: ``Il mio problema è che
sono attaccato al mio corpo. Non dovrei esserlo. È un male, non è vero?
Se fossi una persona saggia non avrei questo attaccamento''. Questo è di
nuovo iniziare partendo da un ideale. È come se cercassimo di scalare un
albero cominciando dalla cima, dicendo: ``Dovrei essere in cima
all'albero. Non dovrei essere quaggiù''. Ma per quanto ci piacerebbe
essere in cima, dobbiamo accettare umilmente che non è così. Per
cominciare, dobbiamo stare vicini al tronco dell'albero, là dove ci sono
le radici, osservando le cose più grossolane e ordinarie, prima di
poterci cominciare a rapportare con qualcosa sulla vetta dell'albero.

È così che si riflette saggiamente. Non si tratta soltanto di purificare
la mente, attaccandosi poi alla purezza. Non si tratta soltanto di
cercare di raffinare la coscienza in modo da riuscire a indurre degli
stati di elevata concentrazione ogni volta che ne abbiamo voglia, perché
anche i più raffinati stati di coscienza sensoriale sono
insoddisfacenti, dipendono da così tante altre cose. Il Nibbāna non
dipende da nessun'altra condizione. Le condizioni di qualsiasi sorta,
che siano brutte, sgradevoli, belle, raffinate o altro, sorgono e
cessano, ma non interferiscono con il Nibbāna, con la pace della mente.

Non stiamo respingendo il mondo sensoriale per una forma di avversione,
perché se cerchiamo di annichilire i sensi, cercando di disfarci di ciò
che non ci piace, questa diventa un'altra abitudine che acquisiamo senza
rendercene conto. Ecco perché dobbiamo essere molto pazienti.

La nostra esistenza da esseri umani è un'esistenza da trascorrere in
meditazione. Considerate il resto della vostra vita, anziché questo
ritiro di dieci giorni, come il tempo della meditazione. Potreste
pensare: ``Ho meditato per dieci giorni. Pensavo di essere illuminato ma
quando tornato a casa per qualche motivo non mi sono sentito più
illuminato. Mi piacerebbe tornare e fare un ritiro più lungo in cui mi
possa sentire più illuminato dell'ultima volta. Sarebbe piacevole avere
uno stato di coscienza più elevato''. In effetti, più la tua esperienza
è raffinata, più la vita quotidiana ti potrà sembrare grossolana. Voli
in alto, e poi quando torni alla routine mondana della tua vita in città
è anche peggio di prima, non è così? Dopo aver toccato certe vette, la
vita quotidiana sembra molto più banale, triviale e spiacevole. La via
alla saggezza derivante da retta comprensione non consiste nel seguire
le proprie preferenze verso ciò che è raffinato rispetto a ciò che è
grossolano, ma nel riconoscere che sia la coscienza raffinata che quella
grossolana sono condizioni impermanenti, sono insoddisfacenti -- la loro
natura non ci soddisferà mai -- e sono \emph{anattā}, non sono ciò che
siamo, non ci appartengono.

Dunque quello del Buddha è un insegnamento molto semplice. Cosa potrebbe
essere più semplice di: ``Ciò che nasce deve morire''? Non si tratta di
una nuova e straordinaria scoperta filosofica, è qualcosa che anche i
popoli primitivi e illetterati sanno bene. Non c'è bisogno di andare
all'università per saperlo.

Quando siamo giovani pensiamo: ``Ho ancora davanti a me tanti anni di
giovinezza e felicità''. Se siamo belli pensiamo: ``Sarò giovane e bello
per sempre'', perché sembra così. Se abbiamo vent'anni, ci stiamo
divertendo, le vita è meravigliosa, e qualcuno ci dice: ``Un giorno
morirai'', potremmo pensare: ``Che persona deprimente. Meglio non
invitarlo più a casa''. Non vogliamo pensare alla morte, vogliamo
pensare a quanto sia stupenda la vita, a quanto piacere ne possiamo
ricavare.

Ma in quanto meditanti riflettiamo sul fatto che si invecchia e si
muore. Questo non vuol dire essere ossessionati morbosamente dalla morte
o essere deprimenti, ma considerare l'intero ciclo dell'esistenza. In
questo modo, quando conosciamo quel ciclo, siamo più attenti a come
viviamo. La gente fa cose orribili perché non riflette sulla propria
morte. Non riflettono e valutano con saggezza, si limitano a seguire le
proprie passioni ed emozioni del momento, cercando di ricavarne piacere,
salvo poi sentirsi arrabbiati e depressi se la vita non dà loro ciò che
vogliono.

Riflettete sulla vostra vita e sulla vostra morte, e sui cicli della
natura. Basta che osserviate ciò che vi rallegra e ciò che vi deprime.
Notate come possiamo sentirci molto positivi o molto negativi, come
vogliamo attaccarci alla bellezza, o alle sensazioni piacevoli, o
all'ispirazione. È davvero piacevole sentirsi ispirati, non è vero? ``Il
Buddhismo è la più grande di tutte le religioni'', oppure ``Quando ho
scoperto il Buddha ero così felice, è una scoperta meravigliosa!''.
Quando incontriamo qualche dubbio e siamo un po' depressi, ci mettiamo a
leggere un libro che ci ispiri, e siamo di nuovo carichi. Ma
ricordatevi, anche l'entusiasmo è una condizione impermanente. È come
diventare felici, devi continuare a fare qualcosa per mantenere questo
stato, e quando fai ripetutamente quel qualcosa, dopo un po' non ti
rende più felice. Quanti dolci puoi mangiare? All'inizio ti rendono
felice - e poi ti fanno star male.

Perciò dipendere solo dall'ispirazione religiosa non basta. Se ti
attacchi all'ispirazione, quando ti sarai stufato del Buddhismo mollerai
tutto per andare alla ricerca di qualcosa di nuovo che ti ispiri. È come
attaccarsi all'aspetto romantico delle relazioni: quando scompare
cominci a cercare qualcuno che ti susciti lo stesso tipo di
romanticismo. Anni fa in America incontrai una donna che era stata
sposata sei volte, e aveva solo trentatre anni. Le dissi: ``Si potrebbe
pensare che dopo la terza o la quarta volta avresti imparato. Perché
continui a sposarti?''. Mi rispose: ``È per il romanticismo. Non mi
piace il rovescio della medaglia, ma adoro l'aspetto romantico''. Almeno
è stata onesta, anche se non particolarmente saggia. Il romanticismo è
una condizione che porta alla disillusione.

Romanticismo, ispirazione, eccitazione, avventura: tutte queste cose
arrivano a un apice e poi condizionano i loro opposti, proprio come
un'inspirazione condiziona un'espirazione. Pensate solo se inspiraste
tutto il tempo. È come avere un'avventura romantica dopo l'altra, no?
Quanto a lungo si può inspirare? L'inspirazione condiziona
l'espirazione, entrambe sono necessarie. La nascita condiziona la morte,
la speranza condiziona la disperazione e l'ispirazione condiziona la
disillusione. Perciò, quando ci attacchiamo alla speranza incontreremo
la disperazione. Quando ci attacchiamo all'eccitazione, ne deriverà la
noia. Quando ci attacchiamo al romanticismo, ne scaturiranno
disillusione e divorzio. Quando ci attacchiamo alla vita, ne conseguirà
la morte. Perciò riconoscete che è l'attaccamento a causare la
sofferenza, attaccarsi alle condizioni e pretendere che siano più di ciò
che sono.

Per tante persone una gran parte della vita sembra trascorrere
nell'attesa e nella speranza che accada qualcosa -- aspettarsi e
anticipare un qualche successo o piacere -- oppure nella preoccupazione
e nella paura che qualcosa di doloroso o spiacevole sia in agguato
dietro l'angolo. Magari si spera di incontrare qualcuno da amare
veramente, o di fare qualche esperienza straordinaria, ma aggrapparsi
alla speranza conduce alla disperazione.

Attraverso una saggia riflessione cominciamo a comprendere quali sono le
cose che creano sofferenza nella nostra vita. Vediamo che in realtà
siamo noi i creatori di quella sofferenza. A causa della nostra
ignoranza, non avendo compreso con saggezza il mondo sensoriale e le sue
limitazioni, ci siamo identificati con tutto ciò che è insoddisfacente e
impermanente, e che può portarci solo all'afflizione e alla morte. Non
c'è da stupirsi che la vita sia così deprimente! Lo è a causa
dell'attaccamento, perché ci identifichiamo e ci rispecchiamo in tutto
ciò che per sua natura è \emph{dukkha}: insoddisfacente e imperfetto.
Ora, quando smettiamo di farlo, quando lasciamo andare, questa è
illuminazione. Siamo esseri illuminati, non attaccati più a nulla, non
identificati più con nulla, non più tratti in inganno dal mondo dei
sensi. Comprendiamo il mondo sensoriale, sappiamo come coesisterci.
Sappiamo come usare il mondo dei sensi per un agire compassionevole, per
un dare gioioso. Non pretendiamo più che sia qui per soddisfarci, per
farci sentire tranquilli e al sicuro o per darci qualcosa, perché non
appena pretendiamo che ci debba soddisfare ne ricaviamo solo afflizione.

Quando non ci identifichiamo più con il mondo sensoriale in quanto
``me'' o ``mio'', e lo vediamo come \emph{anattā}, possiamo godere dei
sensi senza ricercarne l'impatto o dipendere da questo. Non pretendiamo
più che le condizioni siano qualcosa di diverso da ciò che sono, così
quando cambiano possiamo sopportare pazientemente e serenamente il lato
spiacevole dell'esistenza. Possiamo umilmente sopportare malattia,
dolore, freddo, fame, fallimenti e critiche. Se non siamo attaccati al
mondo possiamo adattarci al cambiamento, qualunque esso sia, che sia per
il meglio o per il peggio. Se siamo ancora attaccati non possiamo
adattarci facilmente; stiamo sempre lottando, resistendo, cercando di
controllare e manipolare tutto, e poi ci sentiamo frustrati, spaventati
o depressi in presenza di questo mondo così ingannevole e pauroso. Se
non hai mai contemplato veramente il mondo, se non ti sei mai dato il
tempo per conoscerlo e comprenderlo, questo stesso mondo diventa un
luogo spaventoso. Diventa come una giungla: non sai cosa c'è dietro il
prossimo albero, cespuglio o dirupo: un animale selvaggio, una tigre
feroce mangiatrice di uomini, un terribile drago o un serpente velenoso.

Nibbāna vuol dire venire via dalla giungla. Quando ci orientiamo verso
il Nibbāna stiamo muovendo verso la pace della mente. Benché le
condizioni della mente possano non essere affatto serene, la mente
stessa è sempre un luogo in pace. Qui stiamo facendo una distinzione tra
la mente e le condizioni della mente. Le condizioni della mente possono
essere felici, afflitte, esultanti, depresse, amorevoli o piene d'odio,
preoccupate o in preda alla paura, dubbiose o annoiate. Vanno e vengono
nella mente, ma la mente stessa, come lo spazio in questa stanza, rimane
esattamente così com'è. Lo spazio in questa stanza non ha una qualità
che esalti o deprima, giusto? È esattamente così com'è. Per concentrarci
sullo spazio nella stanza dobbiamo distogliere l'attenzione dalle cose
che vi si trovano. Se ci concentriamo sulle cose presenti nella stanza
diventiamo felici o infelici. Diciamo: ``Guarda che bell'immagine di
Buddha'', oppure se vediamo una cosa che pensiamo sia brutta diciamo:
``Oh, che cosa orribile e disgustosa''. Possiamo passare il tempo a
guardare la gente nella stanza, pensando se questa o quell'altra persona
ci piace oppure no. Possiamo formarci un'opinione sul fatto che le
persone siano in questo o in quest'altro modo, ricordare cosa ci hanno
fatto in passato, congetturare su cosa faranno in futuro, vedere gli
altri come possibili fonti di dolore o gratificazione per noi. Tuttavia,
se distogliamo da loro la nostra attenzione questo non vuol dire che
dobbiamo scacciare tutti dalla stanza. Se non ci concentriamo su nessuna
delle condizioni o non ce ne facciamo assorbire, allora abbiamo una
prospettiva, perché lo spazio nella stanza non ha una qualità che esalti
o deprima. Lo spazio può contenerci tutti quanti, tutte le condizioni
possono andare e venire al suo interno.

Spostandoci dentro di noi, possiamo applicare tutto questo alla mente.
La mente è come uno spazio, c'è posto per tutto oppure niente. Non ha
veramente importanza se è piena o non contiene nulla, perché una volta
che conosciamo lo spazio della mente, la sua vacuità, abbiamo sempre la
giusta prospettiva. Nella mente possono arrivare e poi andarsene
eserciti oppure farfalle, nuvole cariche di pioggia oppure niente. Ogni
cosa può venire e attraversarla, senza che noi cadiamo nella trappola di
una cieca reazione, di una disperata resistenza, del controllo o della
manipolazione.

Così quando dimoriamo nella vacuità delle nostre menti stiamo mettendo
in atto una nuova prospettiva. Non ci stiamo disfacendo delle cose, ma
non ci stiamo più facendo assorbire nelle condizioni che esistono nel
presente, e non ne stiamo creando delle nuove. Questa è la nostra
pratica del lasciar andare. Lasciamo andare la nostra identificazione
con le condizioni, vedendo che sono tutte impermanenti e non-sé. Questo
è quello che intendiamo con meditazione \emph{vipassanā}. Vuol dire
guardare davvero, essere testimoni, ascoltare, osservare che qualunque
cosa viene deve andare. Che sia grossolana o raffinata, buona o cattiva,
qualunque cosa che viene e va non è ciò che siamo. Non siamo buoni, non
siamo cattivi, non siamo maschi o femmine, belli o brutti. Queste sono
le condizioni mutevoli presenti in natura, che sono non-sé. Questa è la
via buddhista all'illuminazione: orientarsi verso il Nibbāna, orientarsi
verso la spaziosità o la vacuità della mente invece di essere preda
delle condizioni.

Ora potreste chiedere: ``Beh, se io non sono le condizioni della mente,
se non sono un uomo o una donna, questo o quello, allora cosa sono?''.
Volete che vi dica chi siete? Mi credereste se lo facessi? Cosa
pensereste se corressi fuori e cominciassi a chiedere in giro ``chi
sono''? È come cercare di vedere i propri occhi: non puoi conoscere te
stesso, perché sei te stesso. Puoi solo conoscere ciò che non sei -- e
questo risolve il problema, non è vero? Se conosci ciò che non sei,
allora non si solleva neppure la questione di chi tu sia. Se dicessi:
``Chi sono? Sto cercando di trovarmi'', e cominciassi a cercare sotto
l'altare, sotto il tappeto, sotto la tenda, pensereste: ``Il Venerabile
Sumedho è andato proprio fuori di testa, è diventato matto, sta cercando
se stesso''. ``Sto cercando me stesso, dove sono?'' è la domanda più
stupida che ci sia. Il problema non è chi siamo, ma il nostro credere e
identificarci in ciò che non siamo. È lì la sofferenza, è lì che
proviamo infelicità, depressione e afflizione. È la nostra
identificazione con tutto ciò che non siamo che è \emph{dukkha}. Quando
ti identifichi con ciò che è insoddisfacente, sarai insoddisfatto e
scontento... è ovvio, no?

Così il cammino del buddhista è un lasciar andare, piuttosto che un
andare alla ricerca di qualcosa. Il problema è il cieco attaccamento, la
cieca identificazione con l'apparenza del mondo sensoriale. Non c'è
bisogno di disfarsi del mondo sensoriale, ma occorre imparare da esso e
osservarlo, senza permettergli più di trarci in inganno. Continuare a
penetrarlo con la saggezza-di-Buddha, continuare a usare questa
saggezza-di-Buddha in modo da trovarsi sempre più a proprio agio con
l'essere saggi, invece di cercare di diventare saggi. Semplicemente
ascoltando, osservando, essendo vigili, essendo consapevoli, la saggezza
emergerà con sempre maggiore chiarezza. Vi troverete a usare la saggezza
in relazione al corpo, in relazione ai pensieri, sentimenti, ricordi,
emozioni, in relazione a tutte queste cose. Le vedrete e ne sarete
testimoni, permettendo che vi attraversino, e lasciandole andare.

E a questo punto non avrete altro da fare che essere saggi, momento per
momento.