\chapter{Ānāpānasati}

Tendiamo a non accorgerci dell'ordinario. Siamo consapevoli del nostro
respiro solo quando è diverso dal solito, per esempio se abbiamo l'asma
o abbiamo fatto una lunga corsa. Ma con \emph{ānāpānasati} prendiamo il
respiro ordinario come oggetto di meditazione. Non cerchiamo di
prolungarlo o accorciarlo, o di controllarlo in qualche modo, ma
rimaniamo semplicemente con la normale inspirazione e la normale
espirazione. Il respiro non è qualcosa che creiamo o immaginiamo, è un
processo naturale dei nostri corpi che continua finché c'è vita, che ci
si concentri su di esso oppure no. Dunque è un oggetto sempre presente;
possiamo rivolgergli la nostra attenzione in qualsiasi momento. Non
abbiamo bisogno di qualifiche speciali per osservarlo. Non abbiamo
nemmeno bisogno di essere particolarmente intelligenti -- tutto ciò che
dobbiamo fare è essere appagati dal respiro, essere consapevoli di una
inspirazione e di una espirazione. La saggezza non scaturisce dallo
studio di grandi teorie e filosofie, ma dall'osservazione
dell'ordinario.

Il respiro è privo di qualità particolarmente eccitanti o avvincenti, e
questo può causarci irrequietezza e avversione. Desideriamo sempre
``ottenere'' qualcosa, trovando qualcosa che ci interessi e ci assorba
senza alcuno sforzo da parte nostra. Se ascoltiamo della musica, non
pensiamo: ``Devo concentrarmi su questa musica ritmica emozionante ed
eccitante'': è che non possiamo farne a meno, perché il ritmo è così
coinvolgente da catturarci completamente. Il ritmo della nostra normale
respirazione non è né interessante né coinvolgente: è tranquillizzante,
e la maggior parte delle persone non è abituata alla tranquillità. Quasi
tutti sono attratti dall'idea della pace, salvo poi trovare l'effettiva
esperienza deludente o frustrante. Vogliono uno stimolo, qualcosa che li
seduca. Con \emph{ānāpānasati} stiamo con un oggetto che è piuttosto
neutro -- non abbiamo forti sentimenti di apprezzamento o antipatia per
il nostro respiro -- e notiamo semplicemente l'inizio di
un'inspirazione, la sua parte centrale e la sua fine, poi l'inizio di
un'espirazione, la sua parte centrale e la sua fine. Il ritmo gentile
del respiro, essendo più lento del ritmo del pensiero, induce alla
tranquillità; cominciamo a smettere di pensare. Non cerchiamo però di
ottenere qualcosa dalla meditazione, di raggiungere il \emph{samādhi}~o
i \emph{jhāna}. Se la mente, infatti, cerca di raggiungere o conseguire
qualcosa, invece di contentarsi umilmente di un respiro, non riesce poi
a rallentare e a diventare calma, e ci sentiamo frustrati.

All'inizio la mente divaga. Quando ce ne accorgiamo, allora con molta
gentilezza la riportiamo sul respiro. Ci poniamo con un atteggiamento di
infinita pazienza e di disponibilità a ricominciare sempre. Le nostre
menti non sono abituate a essere tenute a freno; sono state educate ad
associare una cosa all'altra e a formarsi opinioni su tutto. Poiché
siamo abituati a usare il nostro ingegno e la nostra capacità di pensare
in modi intelligenti, tendiamo a diventare molto tesi e irrequieti se
non lo possiamo fare, e quando pratichiamo \emph{ānāpānasati} proviamo
resistenza e fastidio. Siamo come un cavallo selvaggio a cui sono stati
messi per la prima volta i finimenti, e che si ribella contro ciò che lo
imbriglia.

Quando la mente divaga, ci irritiamo e scoraggiamo, e proviamo
negatività, avversione nei confronti di tutta la faccenda. Se, in preda
alla frustrazione, cerchiamo con la volontà di forzare la mente a
diventare tranquilla, riusciamo a conservare questa condizione solo per
poco e poi la mente riparte in un'altra direzione. L'atteggiamento
giusto nei confronti di \emph{ānāpānasati} è dunque di grande pazienza,
come se disponessimo di tutto il tempo del mondo, lasciando andare o
mettendo da parte tutti i problemi mondani, personali o finanziari. In
questo lasso di tempo non c'è nient'altro che dobbiamo fare tranne
osservare il nostro respiro.

Se la mente divaga durante l'inspirazione, allora mettete più impegno in
questa fase. Se la mente divaga durante l'espirazione, allora è su
questa che metterete più impegno. Continuate a riportarla indietro.
Siate sempre pronti a ricominciare da capo. All'inizio di ogni giornata,
all'inizio di ogni inspirazione, coltivate la mente del principiante,
senza portare niente di vecchio nel nuovo, senza lasciare tracce, come
in un grande falò.

Una inspirazione e la mente divaga, allora la riportiamo indietro -- e
questo è già di per sé un momento di presenza mentale. Stiamo
addestrando la mente come una brava mamma educa il suo bambino. Un bimbo
piccolo non sa cosa sta facendo; si allontana senza farci caso, e se la
mamma si arrabbia con lui, lo sculaccia e lo picchia, il bambino si
spaventa e diventa nevrotico. Una brava mamma lascia libero il bambino
tenendolo d'occhio, e se si allontana lo riporta indietro. Con quella
stessa pazienza, non stiamo cercando di ``fustigarci'', odiando noi
stessi, il nostro respiro, odiando tutti quanti, arrabbiandoci perché
non riusciamo a trovare la tranquillità con \emph{ānāpānasati}.

A volte diventiamo troppo seri su ogni cosa; totalmente privi di gioia,
contentezza o senso dell'umorismo, reprimiamo sempre tutto. Rallegrate
la mente, fatela sorridere. Siate rilassati e a vostro agio, senza la
pressione di dover ottenere qualcosa di speciale. Non c'è niente da
conseguire, niente di eccezionale, niente di speciale. Cosa potete dire
di aver fatto oggi per guadagnarvi la giornata? Solo un'inspirazione
consapevole? Assurdo! Ma è più di quanto la maggior parte della gente
può dire di aver fatto.

Non stiamo combattendo le forze del male. Se vi sentite maldisposti
verso \emph{ānāpānasati}, notate anche questo. Non vivetelo come
qualcosa che dovete fare, ma lasciate che sia una gioia, qualcosa che vi
piace davvero. Quando pensate ``Non ci riesco'', riconoscete che si
tratta di resistenza, paura o frustrazione, e poi rilassatevi. Non
trasformate questa pratica in qualcosa di difficile, in un compito
gravoso. I primi tempi dopo l'ordinazione ero mortalmente serio, con un
atteggiamento arcigno e rigido, come un vecchio palo rinsecchito, e mi
capitava di ritrovarmi in stati mentali tremendi, pensando: ``Devo
riuscirci\ldots{} devo riuscirci\ldots{}''. Fu allora che imparai a
contemplare la pace. Dubbio e irrequietezza, scontento e avversione, ma
presto fui in grado di riflettere sulla pace, ripetendo ipnoticamente
questa parola fino ad autorilassarmi. Cominciavano i dubbi su me stesso
-- ``Non sto concludendo niente, è inutile, voglio però ottenere
qualcosa'' -- ma ero in grado di sentirmi in pace con tutto ciò. Questo
è un metodo che potete utilizzare. Così, quando siamo tesi ci
rilassiamo, e poi riprendiamo \emph{ānāpānasati}.

All'inizio ci sentiamo terribilmente goffi, come quando si impara a
suonare la chitarra. Quando cominciamo a suonare le nostre dita sono
così legate che sembriamo un caso senza speranza, ma dopo che lo abbiamo
fatto per un po' diventiamo più abili ed è piuttosto facile. Stiamo
imparando a osservare ciò che accade nella nostra mente, così ci
accorgiamo quando diventiamo irrequieti e tesi, oppure apatici. Lo
riconosciamo: non cerchiamo di convincerci che le cose stanno
diversamente, siamo pienamente consapevoli di come sono per davvero.
Sosteniamo l'impegno per un'inspirazione. Se non siamo in grado, allora
lo sosteniamo almeno per mezza inspirazione. In questo modo non
cerchiamo di diventare subito perfetti. Non siamo costretti a fare tutto
giusto secondo un'idea preconcetta di come le cose dovrebbero andare, ma
lavoriamo con i problemi che ci sono. Se abbiamo una mente distratta,
allora è saggezza riconoscere che la mente va di qua e di là: questa è
retta comprensione. Pensare che non dovremmo essere così, odiarci o
scoraggiarci perché capita che siamo fatti in questo modo: questa è
ignoranza.

Non cominciamo come yogi perfetti, non facciamo le posizioni di Iyengar
prima di riuscire a fletterci e toccarci le dita dei piedi, sarebbe il
modo più sicuro per farci male. Possiamo anche guardare tutte le
posizioni del libro \emph{Teoria e pratica dello Yoga}, e vedere Iyengar
che si avvolge le gambe intorno al collo nelle posizioni più
straordinarie, ma se provassimo a farle ci porterebbero di corsa in
ospedale. Perciò cominciamo cercando di piegarci un pochino di più dalla
vita, esaminando il dolore e la resistenza che incontriamo, imparando ad
allungarci gradualmente. È lo stesso con \emph{ānāpānasati}:
riconosciamo com'è adesso e partiamo da lì, sosteniamo l'attenzione un
po' più a lungo, e cominciamo a capire cos'è la concentrazione. Non
ripromettetevi prestazioni da Superman se non siete Superman.
Proclamate: ``Starò seduto e osserverò il respiro tutta la notte'', poi
quando non ci riuscite vi arrabbiate. Stabilite dei periodi che siete
sicuri di poter mantenere. Sperimentate, lavorate con la mente fino a
che non saprete come sforzarvi e come rilassarvi.

Dobbiamo imparare a camminare cadendo. Osservate i bambini: non ne ho
mai visto uno che riuscisse a camminare di colpo. I bambini imparano a
camminare gattonando, reggendosi alle cose, cadendo e poi tirandosi su
nuovamente. È lo stesso con la meditazione. Impariamo la saggezza
osservando l'ignoranza, facendo un errore, riflettendo e poi proseguendo
senza mollare. Se ci pensiamo troppo, sembra impossibile. Se i bambini
pensassero molto non imparerebbero mai a camminare, perché quando guardi
un bambino piccolo che cerca di camminare sembra un caso senza speranza,
dico bene? Quando ci pensiamo, la meditazione può apparire completamente
impossibile, ma tiriamo dritto e pratichiamo. È facile quando siamo
pieni di entusiasmo, davvero ispirati dall'insegnante o
dall'insegnamento, ma l'entusiasmo e l'ispirazione sono condizioni
impermanenti, ci conducono alla disillusione e alla noia.

Quando siamo annoiati, dobbiamo davvero mettere ogni impegno nella
pratica. Quando siamo annoiati vogliamo girarci da un'altra parte e
rinascere in qualcosa di affascinante ed eccitante. Ma comprensione
profonda e saggezza richiedono che si debbano sopportare e attraversare
con pazienza i momenti di calo, in cui ci sentiamo disillusi o depressi.
Solo in questo modo possiamo smettere di rafforzare il ciclo
dell'abitudine, e arrivare a comprendere la cessazione, arrivare a
conoscere il silenzio e il vuoto della mente.

Se leggiamo dei libri che suggeriscono di non mettere alcuno sforzo
nella pratica, lasciando semplicemente che tutto avvenga in modo
naturale e spontaneo, potremmo pensare che tutto quel che dobbiamo fare
è stare seduti senza far niente -- salvo poi scivolare in uno stato di
passività e apatia. Nella mia pratica personale mi resi conto che se mi
ritrovavo in uno stato di torpore era importante mettere un certo sforzo
nella postura. Mi accorsi che era del tutto inutile impegnarsi in modo
solo passivo. Quindi mi tiravo su ben eretto, spingevo in avanti il
torace e mettevo energia nella posizione seduta; oppure facevo la
posizione verticale sulla testa o quella della candela sulle spalle.
Anche se nei primi tempi non avevo moltissima energia, riuscivo però lo
stesso a fare qualcosa che richiedesse dello sforzo. Imparavo a
sostenerlo per alcuni secondi e poi lo perdevo di nuovo, ma era sempre
meglio che non fare nulla.

Più imbocchiamo la via facile, la via della minore resistenza, più
seguiamo solo i nostri desideri, più la mente diviene trascurata,
distratta e confusa. È facile pensare, è più facile stare seduti e
pensare tutto il tempo che non pensare: è un'abitudine che abbiamo
acquisito. Anche il pensiero ``Non dovrei pensare'', è solo un altro
pensiero. Per evitare i pensieri dobbiamo esserne consapevoli, dobbiamo
fare lo sforzo di osservare e ascoltare, prestando attenzione al flusso
nelle nostre menti. Invece di pensare alla nostra mente, la osserviamo.
Invece di venire catturati dai pensieri, continuiamo a riconoscerli. Il
pensiero è movimento, è energia, va e viene, non è una condizione
permanente della mente. Quando lo riconosciamo semplicemente come tale,
senza valutarlo o analizzarlo, il pensiero comincia a rallentare e a
fermarsi. Non si tratta di annichilimento: è permettere alle cose di
cessare. È compassione. Man mano che il pensiero ossessivo abituale
inizia a svanire, cominciano ad apparire grandi spazi che nemmeno
sapevamo esistessero.

Assorti nel respiro naturale rallentiamo ogni cosa e calmiamo le
formazioni kammiche. Questo è ciò che intendiamo con \emph{samatha} o
tranquillità: giungere al luogo della calma. La mente diventa
malleabile, agile e flessibile, e il respiro può diventare molto
sottile. Portiamo la pratica di \emph{samatha} solo fino al punto di
\emph{upacāra samādhi} (concentrazione d'accesso), non cerchiamo di
assorbirci completamente nell'oggetto ed entrare nei \emph{jhāna}. A
questo punto siamo ancora consapevoli sia dell'oggetto che del suo
contorno. L'agitazione estrema, nelle sue diverse forme, è diminuita
considerevolmente, ma possiamo ancora operare usando la saggezza.

Con la facoltà della saggezza ancora funzionante, investighiamo, e
questa è \emph{vipassanā} -- osservare attentamente e riconoscere che la
natura di qualsiasi cosa si stia sperimentando è impermanente,
insoddisfacente e impersonale. \emph{Anicca}, \emph{dukkha} e
\emph{anattā} non sono concetti in cui crediamo, ma caratteristiche che
possiamo osservare. Investighiamo l'inizio di un'inspirazione e la sua
fine. Osserviamo cos'è un inizio, non pensando a cosa è ma osservando,
consapevoli con pura attenzione dell'inizio di un'inspirazione e della
sua fine. Il corpo respira completamente da solo: l'inspirazione
condiziona l'espirazione e l'espirazione condiziona l'inspirazione, noi
non possiamo controllare nulla. Il respiro appartiene alla natura, non
appartiene a noi -- è non-sé. Quando vediamo questo, stiamo facendo
\emph{vipassanā}.

Il genere di conoscenza che acquisiamo dalla meditazione buddhista ci
rende umili. Ajahn Chah la definisce la conoscenza del lombrico -- non
ti rende arrogante, non ti fa montare la testa, non ti fa sentire che
sei qualcuno o che hai ottenuto qualcosa. In termini mondani, questa
pratica non sembra molto importante o necessaria. Nessuno scriverà mai
il titolo di giornale ``Alle otto di questa sera il Venerabile Sumedho
ha inspirato''! Ad alcuni sembra che sia molto importante pensare a come
risolvere tutti i problemi del mondo -- come sistemare tutto ciò che non
va e aiutare ogni persona del Terzo Mondo. Rispetto a queste cose
osservare il respiro sembra irrilevante, e la maggior parte della gente
pensa: ``Perché sprecare il tempo facendo questo?'' Ci sono state
persone che si sono confrontate con me in proposito, dicendomi: ``Cosa
fate voi monaci seduti lì tutto il tempo? Cosa state facendo per aiutare
l'umanità? Siete solo degli egoisti, vi aspettate che la gente vi porti
il cibo mentre voi non fate che stare seduti a guardare il respiro.
State scappando dal mondo reale''.

Ma cos'è il mondo reale? Chi sta davvero scappando, e da cosa? Cosa
dovremmo affrontare? Scopriamo che quello che le persone chiamano il
mondo reale è il mondo in cui credono, il mondo verso il quale hanno
degli obblighi, oppure il mondo che conoscono e che è loro familiare. Ma
quel mondo è una condizione della mente. Meditare vuol dire confrontarsi
per davvero con il mondo reale, riconoscendo e prendendo atto di com'è
veramente, invece di credergli, o di giustificarlo, o di cercare
mentalmente di annichilirlo. Il mondo reale opera secondo lo stesso
schema di sorgere e cessare del respiro. Non stiamo teorizzando circa la
natura delle cose, prendendo in prestito dagli altri delle idee
filosofiche e cercando di usarle per razionalizzare; osservando il
nostro respiro stiamo in realtà osservando la natura per come essa è.
Quando osserviamo il respiro stiamo in realtà osservando la natura;
comprendendo la natura del respiro possiamo comprendere la natura di
tutti i fenomeni condizionati. Se cercassimo di comprendere tutti i
fenomeni condizionati nelle loro infinite varietà, qualità,
caratteristiche temporali, e così via, sarebbe troppo complesso; le
nostre menti non ne avrebbero la capacità. Dobbiamo imparare dalla
semplicità.

Quindi con una mente tranquilla diventiamo consapevoli dello schema
ciclico, vediamo che tutto ciò che sorge cessa. Questo è il ciclo che
prende il nome di saṃsāra, la ruota di nascita e morte. Osserviamo il
ciclo ``samsarico'' del respiro. Inspiriamo e poi espiriamo: non
possiamo avere solo inspirazioni o solo espirazioni, l'una condiziona
l'altra. Sarebbe assurdo pensare: ``Voglio solo inspirare. Non voglio
espirare. Rinuncio all'espirazione. La mia vita sarà solo una costante
inspirazione''. Sarebbe assolutamente ridicolo. Se vi dicessi così
pensereste che sono matto; ma questo è proprio ciò che fa la maggior
parte della gente. La gente è proprio sciocca quando vuole rimanere
attaccata solo all'eccitazione, al piacere, alla giovinezza, alla
bellezza e al vigore. ``Voglio solo cose belle e non voglio avere niente
a che fare con ciò che è brutto. Voglio piacere, godimento e creatività,
ma non voglio noia o depressione''. È talmente assurdo, come se mi
sentiste dire: ``Non sopporto le inspirazioni. Non ne farò più''. Quando
osserviamo che l'attaccamento alla bellezza, al piacere sensoriale e
all'amore porta sempre all'afflizione, allora il nostro atteggiamento
diventa di distacco. In questo non c'è annichilimento o desiderio di
distruggere, ma semplicemente lasciar andare, non-attaccamento. Non
cerchiamo la perfezione in nessuna parte del ciclo, ma vediamo che la
perfezione è insita nel ciclo nel suo insieme, includendo la vecchiaia,
la malattia e la morte. Ciò che ha origine nell'increato raggiunge il
suo apice e poi ritorna all'increato, e questa è perfezione.

Quando cominciamo a vedere che tutti i saṅkhāra hanno questo schema di
sorgere e cessare, iniziamo a volgerci al nostro interno, verso
l'incondizionato, la pace della mente, il suo silenzio. Iniziamo a fare
l'esperienza di suññatā, il vuoto, che non è l'oblio o il nulla, ma una
quiete chiara e vibrante. Possiamo anzi volgerci verso il vuoto,
piuttosto che verso le condizioni del respiro e della mente. Allora
abbiamo una prospettiva sulle condizioni, e non vi reagiamo più
ciecamente.

C'è il condizionato, l'incondizionato e il conoscere. Cos'è il
conoscere? È memoria? Consapevolezza? È ``me''? Non sono mai riuscito a
scoprirlo, ma posso essere consapevole. Nella meditazione buddhista
stiamo con il conoscere: essendo consapevoli, essendo svegli, essendo
Buddha nel presente, sapendo che tutto ciò che sorge cessa ed è non-sé.
Applichiamo questo conoscere a tutto, sia il condizionato che
l'incondizionato. È trascendere, essendo svegli invece che cercando di
scappare, e tutto questo nella nostra attività quotidiana. Abbiamo le
quattro normali posture - seduti, in piedi, camminando, distesi -- non
c'è bisogno di stare sulla testa, o di fare un salto mortale
all'indietro, o chissà cos'altro. Usiamo le quattro posture normali e il
respiro ordinario, perché ci stiamo dirigendo verso quanto vi è di più
ordinario, l'incondizionato. Le condizioni sono straordinarie, ma la
pace della mente, l'incondizionato, è così ordinario che nessuno lo nota
mai. È sempre presente, ma non ce ne accorgiamo mai perché siamo
attaccati a ciò che è misterioso e affascinante. Veniamo presi dalle
cose che sorgono e cessano, le cose che stimolano e deprimono. Veniamo
catturati da come le cose appaiono -- e dimentichiamo. Ma ora,
meditando, stiamo ritornando a quella fonte, a quella pace, in quella
posizione del conoscere. Allora il mondo viene compreso per quello che
è, e non ne veniamo più tratti in inganno.

La realizzazione del saṃsāra è la condizione per il Nibbāna.
Riconoscendo i cicli dell'abitudine e non essendo più fuorviati né da
loro né dalle loro qualità, realizziamo il Nibbāna. La
conoscenza-di-Buddha riguarda solo due cose: il condizionato e
l'incondizionato. È una comprensione immediata di come sono le cose
proprio ora, senza avidità o attaccamento. In quel momento possiamo
essere consapevoli delle condizioni della mente, delle sensazioni
corporee, di ciò che stiamo vedendo, udendo, gustando, toccando,
odorando e pensando, e anche del vuoto della mente. Il condizionato e
l'incondizionato sono ciò che possiamo realizzare.

L'insegnamento del Buddha è dunque un insegnamento molto diretto. La nostra
pratica non è ``diventare illuminati'', ma essere nel conoscere, adesso.
